\documentclass{article}
\usepackage{verbatim,a4wide}
\title{Eur.~J.~Comb. 240426: Response to referees}
\author{Jean Cardinal \and Vincent Pilaud}
\begin{document}
\maketitle

The authors warmly thank the referees for their careful reading and numerous insightful comments. We consider each of them below.

\section{Report X}

\begin{verbatim}
a question out of curiosity: is there an explicit formula (in each case, and 
also for the associahedron) for a direction under which the poset is increasing, 
for the given geometric realization? (if there is an answer it could be 
mentioned)
\end{verbatim}

\begin{verbatim}
p2 in Theorem 1 it should be mentioned that "node i" means the ith node in 
infix order
\end{verbatim}

Done.

\begin{verbatim}
bottom p2 "describe the geometry", should it not be "provide an explicit 
geometric realization" ? (since the geometric realization is not unique)
\end{verbatim}

?

\begin{verbatim}
p3 the recursive definition of "above" is not ideal. Maybe cut it into 
defining "directly above" and then "above" (recursively using "directly 
above")
\end{verbatim}

?

\begin{verbatim}
p4 when citing Figure 6 it could be mentioned (e.g. in a footnote) that 
even if for n=4 the strong rectangulotope is the permutohedron 
(combinatorially), the geometric realization is not the same (I am not 
very familiar with quotientopes but am a bit amazed by this fact, is it 
that if you remove deletable hyperplanes from the geometric realization 
of the permutohedron you would have a geometric realization of the strong 
rectangulotope with more involved vertex coordinates ?)
\end{verbatim}

Added a remark?

\begin{verbatim}
p7 Figure 7 caption, you should mention if it is for weak of strong 
(or both), and what the gray areas mean
\end{verbatim}

J: These are the flips in strong rectangulations.

\begin{verbatim}
p8 l2 "there is no such formula" -> "there is no known such formula" 
(or is it sure no such formula can exist?)
\end{verbatim}

Done.

\begin{verbatim}
p8 it could be mentioned when defining S(R) and T(R) that they only 
depend (?) on the weak class of R
\end{verbatim}

Done.

\begin{verbatim}
p8 Theorem 2, it is a bit confusing to have notation S with two meanings, 
maybe it could be changed for the subset notation in (ii) (and also other 
instances)
\end{verbatim}

Done.

\begin{verbatim}
p10 first paragraph of 1.6: definition of the small inline symbol, the text 
seems to allow for the top edge of r_i to be aligned with the top edge of 
r_j, and bottom edge of r_i to be aligned with bottom edge of r_j, which 
is not what the drawing suggests
\end{verbatim}

J: I think this is OK as is.

\begin{verbatim}
For theorem 3 you could comment on what quantities in the vertex coordinates 
formula depends on the strong type and not just on the weak type (it seems 
to be only the Iverson bracket?)
\end{verbatim}

J: Good point. The number of common leaves can also change, right?

\begin{verbatim}
p10 l.-2 when you say "quadratic in n" do you mean that the ith coordinate 
is O(n^2) or just that the sum is over a quadratic set of indices ?
\end{verbatim}

J: at least the second.

\begin{verbatim}
p12 second line of paragraph starting with "An arc...", b should be b' in 
"a'\leq a<b\leq b" 
\end{verbatim}

Done.

\begin{verbatim}
p12 "It has a ray" is a bit vague, maybe give more visual intuition on what 
is a ray (is it a codimension 1 cone? or a union of some?)
\end{verbatim}

We added a remark.

\begin{verbatim}
p13 l4 x_i\leq x_i should be x_i\leq x_j
\end{verbatim}

Done.

\begin{verbatim}
p13 l11 the remark "(in other words, if it is above..." may be too early 
(not sure it follows from something explained before)
\end{verbatim}

?

\begin{verbatim}
p15 "bijection between the facets" -> "bijection between the facets of 
dimension 1" (?) also at the end of Section 4.2
\end{verbatim}

J: No, it is the facets (eg faces of codimension 1) of the cone, as written.

\begin{verbatim}
p16 it could be good to give a short proof for Lemma 6 (e.g. explain which 
translate)
\end{verbatim}

Todo...

\begin{verbatim}
p17 again it would be good to give a short proof of Lemma 10 (or a citation)
\end{verbatim}

?

\begin{verbatim}
p17 in proof of Lemma 11, \prec_s^R whould be \prec_w^R,   and \succ_s 
should be \succ_w^R
\end{verbatim}

Done.

\begin{verbatim}
p19 l1 of 4.3 "of the weak rectangulation congruence" -> "of the strong 
rectangulation congruence"
\end{verbatim}

Done.

\begin{verbatim}
p19 l4 of proof of Lemma 14: "cross" -> "crosses"
\end{verbatim}

Done.

\begin{verbatim}
p19 middle "the arc starts" -> "the arc that starts"
\end{verbatim}

Done.

\begin{verbatim}
p19 bottom "the up and down shard polytopes" -> "the up shard polytopes 
and the down shard polytopes" (same for "each up and down...")
\end{verbatim}

?

\begin{verbatim}
p20 Lemma 20 in the first bullet: "and vertical" should be deleted
\end{verbatim}

Done.

\begin{verbatim}
p21 two lines before Lemma 21 (announcing Lemma 21) you write "which do 
not have any associated yin arc", but this does not appear anymore in the 
statement of Lemma 21 (I do not see well how it relates)
\end{verbatim}

Todo.

\begin{verbatim}
p21 in the proof of Lemma 21 "the right edge of r_i can not be on the 
right of the left edge of r_j" I wonder if it is well defined (it may 
depend on the geometric realization of the strong rectangulation)
\end{verbatim}

Todo.

\begin{verbatim}
after Lemma 21, I am not sure Lemmas 20 and 21 are enough to give the 
first part of Theorem 3, this would be the case if \prec_s was a total order. 
The case of pairs i,j not comparable in the strong order may also have to be 
covered in Lemma 21
\end{verbatim}

?

\begin{verbatim}
p21 after "this concludes the proof of Theorem 3" you could add a remark on 
whether there is an analogue of Theorem 3 for semi-Baxter permutations and 
their corresponding strong rectangulations, this could be a nice corollary
\end{verbatim}

Todo.

\section{Report Y}

\begin{verbatim}
Page 4:

"The strong rectangulation congruence... Its classes are in bijection with 
strong rectangulations.  The corresponding quotientopes will be referred 
to as the strong rectangulotopes and denoted by SR(n). Strong rectangulotopes 
can be obtained as quotientopes [PS19, PPR23] for the strong rectangulation 
congruence." 

The last sentence seems to just repeat what has already been 
said earlier in the paragraph (except that it adds important references).  
Perhaps this paragraph can be revised.
\end{verbatim}

Todo.

\begin{verbatim}
Page 8:

In the definition of "inorder", the word "larger" appears twice and one of 
the instances should be "smaller".  ("...smaller than all labels in the right 
subtree...")
\end{verbatim}

Done.

\begin{verbatim}
The definition of horizontal and vertical subtrees could be made clearer, 
especially since it will be applied to two different trees that appear in 
two orientations.  I assume that the horizontal subtree for i is the subtree 
that is reached from i by a horizontal edge, but it would be good to say 
that explicitly (or correct my misunderstanding).
\end{verbatim}

Done.

\begin{verbatim}
Page 10:

Some readers may appreciate a sentence defining the logical not symbol 
that you use (perhaps in the sentence after defining the Iverson bracket.
\end{verbatim}

Done. (J:?)

\begin{verbatim}
In Theorem 3, there needs to be more clarification on "consecutive intervals".  
At first glance, I wasn't even sure whether that was a description of each 
interval separately or a description of the pair of intervals.  And then it 
wasn't clear whether they could be consecutive in either order or if I is always 
smaller.  After looking deeper into the paper, I conclude that you mean that 
the smallest element of J is one more than the largest element of I.  But you 
should say that explicitly somewhere (or correct my misunderstanding), 
probably right before the statement of Theorem 3.
\end{verbatim}

Done.

\begin{verbatim}
Page 11:

Typo in "...both from a lattice and geometric perspectives".  
Either an "a" or an "s" should be deleted.
\end{verbatim}

Done.

\begin{verbatim}
Page 12:

"It has a chamber for each permutation sigma of [n], given by the set of vectors 
whose coordinates are ordered as sigma."  It's a little unclear what you mean 
here, but as a "public service", it might be worth being very clear and pointing 
out to make these regions to match the vertices of the permutohedron in such 
a way that the edges are *right* weak order cover relations.  (As you know, 
it's *not* the obvious thing, "make the entries of sigma into a vector and 
let sigma label the region that contains that vector", but everyone assumes 
it is.)  Probably you could say something useful in far fewer words than I 
just used.
\end{verbatim}

Todo.

\begin{verbatim}
Page 13:

"An immediate consequence is that all 2n - 2 rays of the braid arrangement 
are preserved in the quotient fan F if and only if the arc ideal A contains 
all up arcs and all down arcs (in other words, if it is above the weak 
rectangulation congruence in the lattice of congruences of the weak Bruhat order)."  
This is a very interesting property of the weak rectangulation congruence.  
I wonder if you should make more of it (perhaps in the introduction and maybe 
also by making it a numbered result). (But also, maybe it belongs later 
in the paper than page 13, after you have said more about this congruence?  
On page 13, you don't yet know the connection between this congruence and up/down arcs.  
Maybe all of this belongs at the end of Section 3.3.)
\end{verbatim}

Todo.

\begin{verbatim}
Page 15:  

There is a missing head on the arrow to 12 on the right drawing of Figure 13.
\end{verbatim}

Done.

\end{document}
